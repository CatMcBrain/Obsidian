\documentclass[10pt]{article}
\usepackage{multicol}
\usepackage{calc}
\usepackage{ifthen}
\usepackage[margin=0.8in]{geometry}
\usepackage{amsmath,amsthm,amsfonts,amssymb}
\usepackage{color,graphicx,overpic}
\usepackage{hyperref}
\usepackage{xcolor}
\usepackage{color}

% Turn off header and footer
\pagestyle{empty}

% Redefine section commands to use less space
\makeatletter
\renewcommand{\section}{\@startsection{section}{1}{0mm}%
                                {-1ex plus -.5ex minus -.2ex}%
                                {0.5ex plus .2ex}%x
                                {\underline\normalfont\large\bfseries}}
\renewcommand{\subsection}{\@startsection{subsection}{2}{0mm}%
                                {-1explus -.5ex minus -.2ex}%
                                {0.5ex plus .2ex}%
                                {\color{blue}\normalfont\normalsize\bfseries}}
\renewcommand{\subsubsection}{\@startsection{subsubsection}{3}{0mm}%
                                {-1ex plus -.5ex minus -.2ex}%
                                {1ex plus .2ex}%
                                {\normalfont\small\bfseries}}
\makeatother

% Define BibTeX command
\def\BibTeX{{\rm B\kern-.05em{\sc i\kern-.025em b}\kern-.08em
    T\kern-.1667em\lower.7ex\hbox{E}\kern-.125emX}}

% Don't print section numbers
\setcounter{secnumdepth}{0}


\setlength{\parindent}{0pt}
\setlength{\parskip}{0pt plus 0.5ex}

%My Environments
\newtheorem{example}[section]{Example}
% -----------------------------------------------------------------------

\begin{document}
\centering
\begin{center}
    \Large{\underline{FISICA II}} \\\vspace{-0.2cm}
    {\small Electricidad}\\
    {\scriptsize Álvaro Jerónimo Sánchez\\}
\end{center}
\begin{multicols}{3}
\raggedright


% multicol parameters
% These lengths are set only within the two main columns
%\setlength{\columnseprule}{0.25pt}
\setlength{\premulticols}{1pt}
\setlength{\postmulticols}{1pt}
\setlength{\multicolsep}{1pt}
\setlength{\columnsep}{2pt}

\section{\underline{Definiciones}}

\subsection{Unidades}

$Q$ (Carga) $[C]$\\
$\vec{E}$ (Campo eléctrico) $[\frac{N}{C}]$ o $[\frac{V}{m}]$\\
$V$ (Potencial eléctrico) $[V]$ o [$\frac{J}{C}$]\\
$\Phi$ (Flujo eléctrico) $[V\, m]$\\
$U_E$ (Energía potencial eléctrica) $[J]$\\
$C$ (Capacidad) $[F]$

\subsection{Geometría}

Circunferencia esfera: $2\pi r$\\
Superficie esfera: $4\pi r^2$\\
Volumen esfera: $\frac{4}{3}\pi r^3$\\
Superficie cilindro: $2\pi rl$\\
Densidad lineal: $\lambda = \frac{Q}{l}$\\
Densidad superficie: $\sigma = \frac{Q}{A}$\\
Densidad volumétrica: $\rho = \frac{Q}{V}$\\
$\vec{u_r} = \frac{1}{r}\vec{r}$

\subsection{Trigonometría}

$sen\theta = \frac{cat_o}{h}$\\
$cos\theta = \frac{cat_a}{h}$\\
$tan\theta = \frac{sen\theta}{cos\theta}$\\
$1 = sen^2\theta + cos^2\theta$


\section{\underline{Básico}}

\subsection{Coulomb}

$\vec{E} = k\frac{Q}{r^2}\vec{u_r}$ (para cargas puntuales)\\
$\vec{F}=k\frac{q_1q_2}{r^2}\vec{u_r}$\\
$V=k\frac{Q}{r}$ ; $k=\frac{1}{4\pi\epsilon_o}$

\subsection{Gauss}

$\Phi = \vec{E}\cdot \vec{S} \rightarrow EScos\theta$\\
$\Phi = \frac{Q_{enc}}{\epsilon_o}$\\
$\Phi = \int \vec{E}\cdot d\vec{s}$

\subsection{Potencial}

\textcolor{gray}{($E_p = \vec{F}(r)d\vec{r} ; \Delta U = -\int_{A}^{B}\vec{F}(r)d\vec{r}$)}\\\vspace{0.1cm}
$\Delta U = -W_{campo}$\\
$W_{\infty 2}=-k\frac{q_1q_2}{r_{12}}$\\
\vspace{0.1cm} \textcolor{red}{$dV=-\vec{E}(r)dr\leftrightarrow \vec{E}(r)=\nabla V(\vec{r})$}

\section{\underline{Dist. contínuas}}

\subsection{Hilo infinito}

$E = \frac{2k\lambda}{d}$ (perpendicular)\\
\vspace{0.2cm}($d$ = distancia hasta "P")

\subsection{Anillo}

$\vec{E}=k\frac{Q_{(a)}}{\sqrt{(a^2+R^2)^3}}\vec{u_a}$\\
$V=k\frac{Q}{\sqrt{a^2+R^2}}$ ($\sqrt{a^2+R^2}=r$)\\
\vspace{0.2cm} ($a$ = distancia hasta "a")

\subsection{Disco}

$\vec{E}=2\pi k\sigma\left(1-\frac{a}{\sqrt{a^2+R^2}} \right)\vec{u_a}$\\
$\rightarrow \vec{E}=\frac{2kQ}{R^2}(...)\vec{u_a}$\\

\subsection{Plano}

$\vec{E} = \frac{\sigma}{2\epsilon_o}$

\subsection{Esfera corteza}
$\boxed{r>R}$\vspace{0.1cm}\\
$\vec{E}=k\frac{Q}{r^2}\vec{u_r}$ ; $V=K\frac{Q}{r}$\\
\vspace{0.2cm} $\boxed{r<R}$ \vspace{0.1cm}\\
$\vec{E}=0$ ; $V=k\frac{Q}{\textcolor{red}{R}}$

\subsection{Esfera homogénea}
$\boxed{r>R}$\vspace{0.1cm}\\
$\vec{E}=k\frac{Q}{r^2}\vec{u_r}$ ; $V = k\frac{Q}{r}$\\\columnbreak
\vspace{0.2cm}$\boxed{r<R}$\vspace{0.1cm}\\
$\vec{E}=k\frac{Qr}{R^3}\vec{u_r}$\\
$V = \frac{3}{2}k\frac{Q}{R}-\frac{1}{2}k\frac{Qr^2}{R^3}$

\subsection{Cilindro}

$\boxed{r>R}$\vspace{0.1cm}\\
$E = \frac{\sigma R}{\epsilon_o r}$\\
\vspace{0.1cm} \textcolor{gray}{(r = distancia a "P")}\\
$\boxed{r=R}$\vspace{0.1cm}\\
$E=\frac{\sigma}{\epsilon_o}$\\
$\boxed{r>R}$\vspace{0.1cm}\\
$E=0$ \textcolor{gray}{$Q=0$}

\section{\underline{Conductores}}

$\vec{E}_{dentro} = 0 \rightarrow Q_{enc} = 0$\\
(Toda $Q$ en superficie)

\subsection{Lámina}

\textcolor{gray}{($E_{dentro} = 0$ ; $ES = \frac{\sigma S}{\epsilon_o}$)}\\
\vspace{0.1cm} $E=\frac{\sigma}{\epsilon_o}$

\subsection{Esf. hueca}

$[r>R]\equiv [r<R]$ (continuidad)

\section{\underline{Condensadores}}

$C=\frac{Q}{V}$\textcolor{gray}{$=\frac{Q}{Q_o/\epsilon_oA=\epsilon_o\frac{A}{d}}$}\\
$U=\frac{1}{2}CV^2$

\subsection{En paralelo}

$Q_1+Q_2=Q$ ; $V_1=V_2=V_o$\\
$C_1+C_2=C$\\
\textcolor{gray}{$U_in = U_f$ (si $V_2 = 0\rightarrow U_f\le U_in$)}

\subsection{En serie}

$Q_1=Q_2=Q$ ; $V_1+V_2=V_o$\\
$\frac{1}{C_1}+\frac{1}{C_2}=\frac{1}{C}$ \textcolor{gray}{$\frac{Q}{C_1}+\frac{Q}{C_2}=\frac{Q}{C}$}

\vfill

\rule{0.3\linewidth}{0.25pt}
\end{multicols}
\end{document}