\documentclass[10pt]{article}
\usepackage{multicol}
\usepackage{calc}
\usepackage{ifthen}
\usepackage[margin=0.8in]{geometry}
\usepackage{amsmath,amsthm,amsfonts,amssymb, mathtools}
\usepackage{color,graphicx,overpic}
\usepackage{hyperref}
\usepackage[dvipsnames]{xcolor}
\usepackage{color}
\usepackage{tcolorbox}
\usepackage{titlesec}
\usepackage{varwidth}

% Turn off header and footer
\pagestyle{empty}

% Redefine section commands to use less space
\makeatletter
\renewcommand{\section}{\@startsection{section}{1}{0mm}%
                                {-1ex plus -.5ex minus -.2ex}%
                                {0.5ex plus .2ex}%x
                                {\normalfont\sffamily\Large\bfseries}}
\renewcommand{\subsection}{\@startsection{subsection}{2}{0mm}%
                                {-1explus -.5ex minus -.2ex}%
                                {0.5ex plus .2ex}%
                                {\color{blue}\normalfont\normalsize\bfseries}}
\renewcommand{\subsubsection}{\@startsection{subsubsection}{3}{0mm}%
                                {-1ex plus -.5ex minus -.2ex}%
                                {1ex plus .2ex}%
                                {\normalfont\small\bfseries}}
\makeatother

% Highlight in math display
\newcommand{\mathcolorbox}[2]{\colorbox{#1}{$\displaystyle #2$}}

% Define BibTeX command
\def\BibTeX{{\rm B\kern-.05em{\sc i\kern-.025em b}\kern-.08em
    T\kern-.1667em\lower.7ex\hbox{E}\kern-.125emX}}

% Don't print section numbers
\setcounter{secnumdepth}{0}


\setlength{\parindent}{0pt}
\setlength{\parskip}{0pt plus 0.5ex}

%My Environments
\newtheorem{example}[section]{Example}

\definecolor{purp}{RGB}{222, 184, 242}

\newcommand{\boxing}[2]{
    \colorbox{#1}{
        \begin{varwidth}{\columnwidth}
            #2
        \end{varwidth}
        }
    }

% -----------------------------------------------------------------------

\begin{document}
\centering

\begin{multicols}{3}
\raggedright


% multicol parameters
% These lengths are set only within the two main columns
%\setlength{\columnseprule}{0.25pt}
\setlength{\premulticols}{1pt}
\setlength{\postmulticols}{1pt}
\setlength{\multicolsep}{1pt}
\setlength{\columnsep}{2pt}

\section{\underline{Definiciones}}

\robustify{\fbox}

\subsection{Unidades}

$Q$ (Carga) $[C]$\\
$\vec{E}$ (Campo eléctrico) $[\frac{N}{C}]$ o $[\frac{V}{m}]$\\
$V$ (Potencial eléctrico) $[V]$ o [$\frac{J}{C}$]\\
$\Phi$ (Flujo eléctrico) $[V\, m]$\\
$U_E$ (Energía potencial eléctrica) $[J]$\\
$C$ (Capacidad) $[F]$\\
\rule{0.3\linewidth}{0.25pt}\\
$I$ (Intensidad) $[A\textcolor{gray}{=C/s}]$\\
$\vec{J}$ (Densidad de corriente) $[A/m^3]$\\
$R$ (Resistencia) $[\Omega]$
$\rho'$ (Resistividad) $[\Omega/m]$\\
$G$ (Conductancia) $[S\textcolor{gray}{=\Omega^{-1}}]$\\
$\sigma'$ (Conductividad) $[m/\Omega\; \textcolor{gray}{= S/m}]$\\
$\mu$ (Movilidad el. de portadores) $[m^2/V]$\\
$n$ (Portadores $/$ unidad de volumen)\\
$\Sigma$ (F. Electromotriz) $[V]$\\
$\vec{\rho}$ (Momento dipolar) $[D\textcolor{gray}{=3.34\cdot 10^{-30}C\cdot m}]$\\
$\chi_e$ (Susceptibilidad eléctrica) $[Adim.]$\\
$\alpha$ (polarizabilidad/densidad) [$m^3$]\\
\rule{0.3\linewidth}{0.25pt}\\
$F_m$ (Fuerza magnética) $[N]$\\
$B$ (Campo magnético) $[T\textcolor{gray}{=\frac{N}{C\; m/s}=\frac{N}{A\; m}}]$\\
$\Phi$ (Flujo de campo mag.) $[Wb]$



\subsection{Geometría}

Superficie círculo:$\pi r^2$\\
Circunferencia esfera: $2\pi r$\\
Superficie esfera: $4\pi r^2$\\
Volumen esfera: $\frac{4}{3}\pi r^3$\\
Superficie cilindro: $2\pi rl$\\
Densidad lineal: $\lambda = \frac{Q}{l}\; dq=\lambda dl$\\
Densidad superficie: $\sigma = \frac{Q}{A}\; dq=\sigma dS$\\
Densidad volumétrica: $\rho = \frac{Q}{V}\; dq=\rho dV$\\
$\vec{u_r} = \frac{1}{r}\vec{r}$

\subsection{Trigonometría}

$sen\theta = \frac{cat_o}{h}$\\
$cos\theta = \frac{cat_a}{h}$\\
$tan\theta = \frac{sen\theta}{cos\theta}$\\
$1 = sen^2\theta + cos^2\theta$

\columnbreak
%%%%%%%%%%%%%%%%%%%%%%%%%%%%%%%%%%

\section{\underline{Básico}}

\subsection{Coulomb}

\boxing{purp}{
{$\vec{E} = k\frac{Q}{r^2}\vec{u_r}$\\
$\vec{F}=k\frac{q_1q_2}{r^2}\vec{u_r}$\\
$V=k\frac{Q}{r}$ ; $k=\frac{1}{4\pi\epsilon_o}$}
}\\
$U=k\frac{q_1q_2}{r}$\\
$W_{\infty }=-k\frac{q_1q_2}{r_{12}}$\\
$\Delta U = -W_{campo}$

\subsection{Gauss}

\boxing{purp}
{$\Phi = \vec{E}\cdot \vec{S} \rightarrow EScos\theta$\\
$\Phi = \frac{Q_{enc}}{\epsilon_o}$}\\
$\Phi = \int \vec{E}\cdot d\vec{s}$

\section{\underline{Dist. contínuas}}

\subsection{Hilo infinito}

$E = \frac{2k\lambda}{d}$ (perpendicular)\\
\vspace{0.2cm}($d$ = distancia hasta "P")

\subsection{Anillo}

$\vec{E}=k\frac{Q_{(a)}}{\sqrt{(a^2+R^2)^3}}\vec{u_a} \xrightarrow{(a\gg R)} \frac{KQ}{a^2}$\\
$V=k\frac{Q}{\sqrt{a^2+R^2}}$ ($\sqrt{a^2+R^2}=r$)\\
\vspace{0.2cm} ($a$ = distancia hasta "a")

\subsection{Disco}

$\vec{E}=2\pi k\sigma\left(1-\frac{a}{\sqrt{a^2+R^2}} \right)\vec{u_a}$\\
$\rightarrow \vec{E}=\tiny\frac{2kQ}{R^2}(...)\vec{u_a}
\left\{\begin{array}{l}
    (R\gg a) \text{=Plano}\\
    (a\gg R) \frac{KQ}{a^2}
\end{array}\right.$

\subsection{Plano}

$\vec{E} = \frac{\sigma}{2\epsilon_o}$

\subsection{Esfera corteza}
$\boxed{r>R}$\vspace{0.1cm}\\
$\vec{E}=k\frac{Q}{r^2}\vec{u_r}$ ; $V=K\frac{Q}{r}$\\
\vspace{0.2cm} $\boxed{r<R}$ \vspace{0.1cm}\\
$\vec{E}=0$ ; $V=k\frac{Q}{\textcolor{red}{R}}$

\subsection{Esfera homogénea}
$\boxed{r>R}$\vspace{0.1cm}\\
$\vec{E}=k\frac{Q}{r^2}\vec{u_r}$ ; $V = k\frac{Q}{r}$\\
\vspace{0.2cm}$\boxed{r<R}$\vspace{0.1cm}\\
$\vec{E}=k\frac{Qr}{R^3}\vec{u_r}$\\
$V = \frac{3}{2}k\frac{Q}{R}-\frac{1}{2}k\frac{Qr^2}{R^3}$

\subsection{Cilindro}

$\boxed{r>R}$\vspace{0.1cm}\\
$E = \frac{\sigma R}{\epsilon_o r}$\\
\vspace{0.1cm} \textcolor{gray}{(r = distancia a "P")}\\
$\boxed{r=R}$\vspace{0.1cm}\\
$E=\frac{\sigma}{\epsilon_o}$\\
$\boxed{r>R}$\vspace{0.1cm}\\
$E=0$ \textcolor{gray}{$Q=0$}

\section{\underline{Conductores}}

$\vec{E}_{dentro} = 0 \rightarrow Q_{enc} = 0$\\
(Toda $Q$ en superficie)

\subsection{Lámina}

\textcolor{gray}{($E_{dentro} = 0$ ; $ES = \frac{\sigma S}{\epsilon_o}$)}\\
\vspace{0.1cm} $E=\frac{\sigma}{\epsilon_o}$

\subsection{Esf. hueca}

$[r>R]\equiv [r<R]$ (continuidad)

%%%%%%%%%%%%%%%%%%%%%%%%%%%%%%%%%%

\section{\underline{Dipolo}}

\boxing{purp}
{$\vec{p}=q\vec{d}$\\
$V=k\frac{q(r_2-r_1)}{r_1r_2}$}\\\vspace{0.1cm}

\textcolor{gray}{\small d=distancia entre polos}\\
{\small \textbf{E sobre eje x:}}\\
$\vec{E_x} = \frac{2xqd}{\left[ x^2 - (d/2)^2 \right]^2}\vec{u_x}\xrightarrow{x\gg \frac{d}{2}} k\frac{2qd}{x^3}\vec{u_x}$\\
{\small \textbf{E sobre eje y:}}\\
$\vec{E}=-2k\frac{q\; d/2}{\left[ y^2+(d/2)^2\right]^{3/2}}\xrightarrow{y\gg \frac{d}{2}}-k\frac{qd}{y^3}\vec{u_x}$

\subsection{Campo homogéneo}

\boxing{purp}
{$M=\vec{p}\times\vec{E_o}$\\
$U=-pE_o$}

\subsection{Polarización Macro}

$\vec{P}=\frac{1}{\Delta Vol}\sum \vec{p_i}\xrightarrow{=dip.}\vec{P}=n\vec{p}$\\
$\vec{P}=\chi_e\epsilon_o\vec{E}=\epsilon_o(\epsilon_r-1)E$\\

\vspace{0.1cm}
\fbox{\textbf{Campos}}\\
\vspace{0.1cm}

$\vec{E}_{dentro}=\vec{E_o}-\vec{E_p}$\\
$E_p=\frac{\sigma_p}{\epsilon_o}$\\
$\epsilon_r=\frac{E_o}{E}=1+\chi_e$\\

\vspace{0.1cm}

$Q_P=\sigma_PA$\\
$p_{tot}=Q_pL$\\
$P=\frac{p_{tot}}{Vol}=\sigma_P$ \textcolor{gray}{$[C/m^2]$}\\
$|\vec{D}|=\epsilon_oE_{int}+P=\epsilon_o\epsilon_rE=\epsilon E$ \textcolor{gray}{($\epsilon=\epsilon_o\epsilon_r$)}

\subsection{Gauss}

\textcolor{red}{Sustituir $\epsilon_o$ en ''$k$'' por \textbf{$\epsilon = \epsilon_o\epsilon_r$}.}

\subsection{Polarización Micro}

\vspace{0.1cm}
\textbf{Polarización elec.}\\
\vspace{0.1cm}

$\vec{p}=\alpha\epsilon_o\vec{E}_{(local)}$\\
$\chi_e=n\alpha=n4\pi R^3$ \textcolor{gray}{(átomo)}\\
$\vec{E}_{nube}=\vec{E_o}\rightarrow E_{nube}=k\frac{qd}{R^3}$\\
\rule{0.6\columnwidth}{0.3pt}\vspace{0.1cm}
$\left.\begin{array}{l}
    p=4\pi\epsilon_o R^3\vec{E_o}\\
    \vec{p}=\alpha\epsilon_o\vec{E_o}
\end{array}\right\}$
$\Rightarrow \alpha_e=4\pi R^3$
\\\textcolor{gray}{\small($R\equiv$ nube $e^-$)}

\vspace{0.1cm}
\textbf{Polarización iónica.}\\
\vspace{0.1cm}

$\vec{p}=\alpha\epsilon_o\vec{E_o}$\\
$\alpha_{\text{orientación}}=\frac{p_o^2}{3\epsilon_ok_BT}$ ; $\chi=n\alpha_{ori.}$\\

\vspace{0.1cm}
\textbf{Interacc. dip.}\\
\vspace{0.1cm}

\fbox{\textbf{Perm. + Perm.}}\\
\vspace{0.1cm}

\textbf{Energía}\\
\vspace{0.1cm}

$U=-\vec{p_2}\vec{E_1}=-p_2E_{1x}$\\
$U=-k\frac{p_1p_2}{r^3}(3\cos^2\theta-1)$\\
\textcolor{gray}{dipolos paralelos}\\
$\Rightarrow$\textbf{Mínima E:} $\pm K\frac{2p_1p_2}{r^3}$\\
\textcolor{gray}{dependiendo de $\theta=\pi/2\, ,\; 0...$}\\
\vspace{0.1cm}
$\frac{dU}{dVol}=\frac{1}{2}\epsilon_r\epsilon_oE^2=\frac{1}{2}\epsilon E^2$ \textcolor{gray}{[$J/m^3$]}\\

\vspace{0.1cm}
\textbf{Campo y potencial}\\
\vspace{0.1cm}

$V=k\frac{\vec{r}\vec{p_1}}{r^3}=k\frac{p_1}{cos\theta}$\\
$E_r=-\frac{dV}{dr}=k\frac{2p\cos\theta}{r^3}$\\
$E_\theta=-\frac{\lambda2v}{r\; d\theta}=k\frac{p\sin\theta}{r^3}$

\vspace{0.1cm}
\fbox{\textbf{Perm. + Ind.}}\\
\vspace{0.1cm}

\textcolor{gray}{\small ($p_1\equiv perm.\; ;\; p_2\equiv ind.$)}\\
$U=-\vec{p_2}E_1$ ; $E_1=E_{1x}=-K\frac{2p_1}{r^3}$\\
$\vec{p_2}=\alpha\epsilon_o\vec{E_1}$\\
$\Rightarrow U=-\frac{k}{\pi}\frac{\alpha p_1}{r^6}=-\frac{C}{r^6}$ 

\vspace{0.1cm}
\fbox{\textbf{Ind. + Ind.}}\\
\vspace{0.1cm}

$U=-\frac{C}{r^6}$\\
$C=\frac{3}{2}\alpha_1\alpha_2\frac{I_1I_2}{I_1+I_2}$\\
\textcolor{gray}{$I_{1,2}\equiv$ E ionización}

%%%%%%%%%%%%%%%%%%%%%%%%%%%%%%%%%%


\section{\underline{Circuitos}}

\subsection{Ohm}

\boxing{purp}
{$I=\frac{V}{R}$\\
$R=\rho'\frac{L}{S}$}\\

\subsection{Corriente}

$I = \frac{\Delta q}{\Delta t} = n\cdot qSv_d$\\
$\Delta Q=n\cdot q\Delta Vol=n\cdot qSv_d\Delta t$\\
$\rightarrow \Delta Vol = Sv_d\Delta t =S\Delta L$\\ \textcolor{gray}{(S = cara sección)}

\subsection{Variables}

$\sigma'=\frac{1}{\rho'}$\\
$G=\frac{1}{R}=\sigma'\frac{S}{L}$\\
$\vec{J}=n\cdot q\vec{v_d} \rightarrow J=\frac{I}{S}$\\
$J=\sigma' E$\\
$\mu = \frac{\sigma'}{n\cdot q}$\\
$v_d=\frac{\sigma'}{n\cdot q}\vec{E}=\mu\vec{E}$\\
\vspace{0.1cm}
\textcolor{gray}{($q$ suele ser del $e^-$)}

\subsection{Energía}

$\Delta U = \Delta Q(V_B-V_A)$\\\small
\textbf{Pérdida E: }\\$-\Delta U=\Delta QV=Q(V_A-V_B)$\\
\textbf{Variac. temp.: }\\$-\frac{\Delta U}{\Delta t}=\frac{\Delta Q}{\Delta t}V=IV$\\
\textbf{Pot disipada: }\\$Pot=I^2R$\\
\vspace{0.1cm}
$Pot(t)=\frac{V_o^2}{R}e^{-t/\tau}=\frac{Q_{max}}{R}e^{-t/\tau}$\\
\vspace{0.1cm}
$\frac{1}{2}m_ev_e^2=qV$

\subsection{Condensadores}

\boxing{purp}
{$C=\frac{Q}{V}$\textcolor{gray}{$=\frac{Q}{Q_o/\epsilon_oA}=\epsilon_o\frac{A}{d}$}\\
$U=\frac{1}{2}CV^2$}\\
\textcolor{gray}{\underline{Vacío}$\rightarrow V_C=\frac{q}{C}=0$}\\
\textcolor{gray}{\underline{Lleno}$\rightarrow I_C=0$}\\\vspace{0.1cm}
\textbf{\small Campo dentro:} $E=E_1+E_2=\frac{\sigma}{\epsilon_o}\; ; \; V=E\cdot d$\\

\vspace{0.1cm}
\fbox{\textbf{Dieléctrico:}}\\
\vspace{0.1cm}

\boxing{purp}
{$C=\epsilon\frac{A}{d}$
$\epsilon_r=1+\chi_e$\\
$\epsilon_r = \frac{\epsilon}{\epsilon_o}$\\
$C_{\epsilon_r}=C_o\cdot\epsilon_r$}\\
$\epsilon_o=\frac{1}{\mu_o c^2}$\\
$k=\frac{1}{4\pi\epsilon_o}$

\subsection{Carga condensador}

$V_o=iR+\frac{q}{C}$\\
$\tau=RC$\\
$q=V_oC\left( 1-e^{-t/\tau} \right)\; \textcolor{gray}{(max=V_oC)}$\\
$i=\frac{dq}{dt}=\frac{V_o}{R}e^{-t/\tau}$\\

\vspace{0.1cm}
\fbox{\textbf{Balance}}\\
\vspace{0.1cm}

\textbf{E$_{aport}$ Batería: }$U_{bat}(t)=V_o^2C\left(1-e^{-t/\tau}\right)$\\
\textbf{E$_{disip}$ Resist.: }$U_{R}(t)=\frac{V_o^2C}{2}\left(1-e^{-2t/\tau}\right)$\\
\textbf{E$_{almac}$ Cond.: }$U_{C}(t)=\frac{q^2}{2C}=\frac{V_o^2C}{2}\left(1-e^{-t/\tau}\right)^2$\\
\textcolor{gray}{$\rightarrow U_{bat}=U_R+U_C$}

\subsection{Descarga condensador}

(\textcolor{gray}{$Q\equiv\; Q_{max}$\tiny con la que empezamos})\\
$q(t)=QE^{-t/\tau}$\\
$i(t)=\frac{dq}{dt}=-\frac{Q}{\tau}e^{-t/\tau}=-\frac{V_o}{R}e^{-t/\tau}$\\

\vspace{0.1cm}
\fbox{\textbf{Balance}}\\
\vspace{0.1cm}

\textbf{E$_{ini}$ Cond.: }$U_{C\; ini}=\frac{Q^2}{2C}$\\
\textbf{E$_{disip}$ Resist.: }$U_R(t)=\frac{Q^2}{2C}\left(1-e^{-2t/\tau} \right)$\\
\textbf{E Cond.: }$U_C(t)=\frac{q^2}{2C} = \frac{Q^2}{2C}e^{-2t/\tau}$\\\vspace{0.1cm}
\textcolor{gray}{$\rightarrow U_{ini}=U_R+U_C$}

\subsection{Asociación}

\vspace{0.1cm}
\fbox{\textbf{En paralelo}}\\
\vspace{0.1cm}

$Q_1+Q_2=Q$ ; $V_1=V_2=V_o$\\
$C_1+C_2=C$\\

\vspace{0.1cm}
\fbox{\textbf{En serie}}\\
\vspace{0.1cm}

$Q_1=Q_2=Q$ ; $V_1+V_2=V_o$\\
$\frac{1}{C_1}+\frac{1}{C_2}=\frac{1}{C}$ \textcolor{gray}{$\frac{Q}{C_1}+\frac{Q}{C_2}=\frac{Q}{C}$}\\

\vspace{0.2cm}
\textcolor{red}{Inverso para \textbf{Resistencias}.}



\section{\underline{Kirchhoff}}

\subsection{Nodos}

$\sum I_i=0$\\
$(I_1\textcolor{PineGreen}{-}I_2\textcolor{PineGreen}{+}I_3...=0)$\\
\textcolor{PineGreen}{Entra $\equiv I>0$, Sale $\equiv I<0$}

\subsection{Mallas}

$\sum\epsilon_i=\sum V_i$\
$(\textcolor{PineGreen}{-}\epsilon_1\textcolor{PineGreen}{+}\epsilon_2...=\textcolor{Magenta}{-}R_1I_1\textcolor{Magenta}{+}R_2I_2...)$\\
\textcolor{PineGreen}{Borne $\oplus \rightarrow \ominus\;\equiv \epsilon>0$}\\
\textcolor{Magenta}{Direcc. $I =$ malla $I>0$}

%%%%%%%%%%%%%%%%%%%%%%%%%%%%%%%%%%

\section{\underline{Magnetismo}}

\textcolor{gray}{(Partícula moviéndose en un B)}

$F_m=q\vec{v}\times\vec{B}$ \textcolor{gray}{($\vec{F_m} \perp \vec{B}\&\vec{v}$)}\\
$\vec{a}=\left\{ \begin{array}{l}
    a_n=v^2/R\\
    a_T=\partial v/\partial t =0
\end{array}\right.$\\
$m\frac{v^2}{R}=qvB\implies R=\frac{mv}{qB}$

\subsection{Selector de velocidades}

$F_m=F_e$\\
$qVB=qE\implies v=\frac{E}{B}$ \textcolor{gray}{(cruzan)}

\subsection{Espectrómetro}

$\left.\begin{array}{l}
    \frac{1}{2}mv^2=qV\\
    R=\frac{mv}{qB}
\end{array}\right\} v=\frac{RqB}{m}$\\$\implies\frac{m}{q}=\frac{B^2R^2}{2V}$

\subsection{F sobre conductor}

\vspace{0.1cm}
\fbox{\textbf{Rectilíneo}}\\
\vspace{0.1cm}

\boxing{purp}
{$\vec{F}=nq\vec{v}SL\times\vec{B}=I\vec{L}\times\vec{B}$}\\

\vspace{0.1cm}
\fbox{\textbf{Espira cuadrada}}\\
\vspace{0.1cm}

\textcolor{gray}{
$F (a\parallel B) = 0$\\
$F (b\perp B)\neq 0$ (momento)\\
($b\leadsto \vec{l}$)
}

\subsection{Ampère}

\boxing{purp}
{$\oint\vec{B}\,\vec{dl}=\mu_oI\xrightarrow{\text{hilo}}B=\frac{\mu_oI}{2\pi R}$}\\
$I=\sum I_i$\\\vspace{0.1cm}
$\frac{\partial F}{\partial l}=\frac{\mu_o}{2\pi}\frac{I_1I_2}{d}$ \textcolor{gray}{(entre corrientes)}

\subsection{Inducción}

\boxing{purp}
{$\Phi_M=\iint_S \vec{B}\,\vec{dS}\xrightarrow[B\parallel S]{B=cte}\Phi=BS$\\
$\Phi_{\text{solenoide}}=N\cdot BS$}

\subsection{Faraday-Lenz}

\textbf{Faraday:}
$fem=\textcolor{gray}{(N)}\frac{\partial\Phi}{\partial t}$\\
\boxing{purp}
{\textbf{F-Lenz:}
$fem=-\textcolor{gray}{(N)}\frac{\partial\Phi}{\partial t}=\frac{-\partial}{\partial t}\int\vec{B}\vec{dS}$}

\subsection{Efecto Hall}

$\left.\begin{array}{l}
    F_E=qE_H\\
    F_m=qvB
\end{array}\right\}
\begin{array}{l}
    E_H=vB\\
    V_H=E_Ha=VBa
\end{array}$\\
$I=nqvS\xrightarrow{S=a\cdot d} v=\frac{I}{nqad}$\\
$\implies V_H=\frac{IB}{nqd}$

\vfill

\end{multicols}
\end{document}