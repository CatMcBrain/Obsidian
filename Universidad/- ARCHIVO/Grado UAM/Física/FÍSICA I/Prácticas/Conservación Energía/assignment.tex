\documentclass[]{scrartcl}

%Packages________________
\usepackage[normalem]{ulem}
\usepackage{cancel, xcolor}
\usepackage{parskip}
\usepackage[Spanish]{babel}
\usepackage{newtxtext}
\usepackage[varvw]{newtxmath}
\usepackage{booktabs, colortbl, bigstrut, multirow, multicol}
\usepackage{url}
\usepackage{pgfplots}
\usepackage{graphicx}
\usepackage{amsmath}
\usepackage{tabularx}
\usepackage{floatrow}
\usepackage{float}
\usepackage{cancel}

%Commands________________
\newcommand\hcancel[2][black]{\setbox0=\hbox{$#2$}%strikeout
\rlap{\raisebox{.45\ht0}{\textcolor{#1}{\rule{\wd0}{1pt}}}}#2} 

\renewcommand\theequation{\Alph{equation}}

%\renewcommand\thesubsection{\Alph{subsection}}

%Titlepage_________________
\title{\vspace{-1.8cm}Conservación de la Energía}
\author{Álvaro Jerónimo Sánchez}
\date{}

%Document_____________
\begin{document}
\maketitle
\setcounter{section}{3}

\section{Cuestiones}

\subsection{Fuerzas que intervienen}

\begin{itemize}
    \item \textbf{Peso:} sí genera trabajo  y es conservativa (sólo depende de la posición).
    \item \textbf{Rozamiento:} sí genera trabajo y no es conservativa (disipa energía en forma de calor).
    \item \textbf{Centrípeta:} no genera trabajo (90°) y es conservativa (en este caso es resultado de la normal generada por el raíl).
    \item \textbf{Normal:} no genera trabajo (no desplaza al objeto) y no es conservativa (es una fuerza de contacto).
\end{itemize}

\subsection{Altura mínima para realizar el loop}

Para superar el loop, la $N$ en el punto más alto de éste tendrá que ser $0$. Aplicando la Segunda Ley de Newton y sustituyendo con las fuerzas que influencian a la esfera, podemos hallar la velocidad mínima a la que tiene que ir para superar el loop:

\begin{gather*}
    \sum F=ma\\
    mg+\cancelto{0}{N}=m\cdot a_{n} = m\frac{v^{2}}{R}\\
    v_{min}=\sqrt{gR}
\end{gather*}

Aplicando la conservación de la energía mecánica, siendo $A$ el punto en el que comienza la esfera y $B$ el punto más alto del loop (2R):

\begin{gather*}
    E_{m(A)}=E_{m(B)}\\
    \cancelto{0}{E_{c}(A)} + mgh_{min} = \frac{1}{2}mv^{2}_{min} + mg2R\\
    gh_{min}=\frac{1}{2}gR+2gR \rightarrow gh_{min}=gR(\frac{1}{2}+2)\\
    h_{min}=R\frac{5}{2} = 9\cdot\frac{5}{2} = \boxed{22.5 cm}
\end{gather*}

La altura hallada experimentalmente es mayor que la esperada usando el principio de la conservación de la energía mecánica, y esto se puede deber a factores que no se han tenido en cuenta como la resistencia del aire o desperfectos de la superficie que hacen que la esfera tenga rozamiento extra.

Para hallar el valor de la fuerza normal en la parte inferior del loop, primero despejo la velocidad en este punto ($C$):

\begin{gather*}
    \frac{1}{2}mgR+mg2R=0+\frac{1}{2}mv^{2}_{C}\\
    v^2_{C}=\sqrt{5gR}
\end{gather*}

Aplicando la Segunda Ley de Newton nuevamente, hallo que la normal en el punto más bajo es:

\begin{gather*}
    -mg+N=ma_{n}=m\frac{v^{2}}{R}\\
    -mg+N=m\frac{5gR}{r} \rightarrow N_{C}=6mg = \boxed{0.94 N}
\end{gather*}

\subsection{Comparación con el bloque}

Para que un bloque realizase el loop, la altura necesaria sería mucho mayor, ya que más superficie del cuerpo está en contacto con la pista y la fuerza de rozamiento que éste sufre es notablemente mayor que la de la esfera.

\subsection{Trabajo al dejarlo caer desde 63cm}

Teniendo en cuenta que en ambos puntos la energía cinética será 0, el trabajo realizado por la fricción será:

\begin{gather*}
    W'=\cancelto{0}{\Delta E_{c}} + \Delta E_{p} = mg(h_{2}-h_{1})\\
    W'=0.004g(0.23-0.63) = \boxed{-0.02 N}
\end{gather*}

Y la energía es invertida en forma de calor.

\end{document}